\documentclass[11pt]{exam}

\usepackage{amssymb, amsmath, amsthm, mathrsfs, multicol, graphicx}
\usepackage{tikz, pgfplots}


\def\d{\displaystyle}
\def\?{\reflectbox{?}}
\def\b#1{\mathbf{#1}}
\def\f#1{\mathfrak #1}
\def\c#1{\mathcal #1}
\def\s#1{\mathscr #1}
\def\r#1{\mathrm{#1}}
\def\N{\mathbb N}
\def\Z{\mathbb Z}
\def\Q{\mathbb Q}
\def\R{\mathbb R}
\def\C{\mathbb C}
\def\F{\mathbb F}
\def\A{\mathbb A}
\def\X{\mathbb X}
\def\E{\mathbb E}
\def\O{\mathbb O}
\def\pow{\mathscr P}
\def\inv{^{-1}}
\def\nrml{\triangleleft}
\def\st{:}
\def\~{\widetilde}
\def\rem{\mathcal R}
\def\iff{\leftrightarrow}
\def\Iff{\Leftrightarrow}
\def\and{\wedge}
\def\And{\bigwedge}
\def\AAnd{\d\bigwedge\mkern-18 mu\bigwedge}
\def\Vee{\bigvee}
\def\VVee{\d\Vee\mkern-18 mu\Vee}
\def\imp{\rightarrow}
\def\Imp{\Rightarrow}
\def\Fi{\Leftarrow}


\def\bar{\overline}

%\pointname{pts}
\pointsinmargin
\marginpointname{pts}
\marginbonuspointname{ bns pts}

\addpoints
\pagestyle{headandfoot}
%\printanswers


\header{MATH 131}{\bf\large Learning Target 15 Quiz}{Fall 2025}
\runningfooter{}{}{Version \version}
\extrafootheight{-.45 in}



\begin{document}
\def\version{A}
%space for name
\noindent {\large\bf Name:} \underline{\hspace{2.5 in}}
\vskip 1em


\begin{questions}
\question The function $f(x)$ (which you don't know) has \emph{first and second derivatives}
\[
f'(x) = e^x(x-4)(x+3)
\]
\[
f''(x) = e^x(x^2 + x - 13)
\]
Using these provided derivatives, find all \underline{critical numbers} of the original function $f(x)$, and then use the first or second derivative tests to classify them as local maximums, local minimums, or neither.  Then give the intervals on which $f$ is increasing or decreasing.  

Use the middle of the page to show your work and record your answers at the bottom of the page.
\vfill
Critical numbers:
\vskip 1em
Local maximum(s) at $x = $
\vskip 1em
Local minimum(s) at $x = $
\vskip 1em
$f$ is increasing on the interval(s):
\vskip 1em
$f$ is decreasing on the interval(s):
\end{questions}



\newpage

\def\version{B}
%space for name
\noindent {\large\bf Name:} \underline{\hspace{2.5 in}}
\vskip 1em
\begin{questions}
\question The function $f(x)$ (which you don't know) has \emph{first and second derivatives}
\[
f'(x) = x(x-1)(x+2)
\]
\[
f''(x) = 3x^2 + 2x - 2
\]
Using these provided derivatives, find all \underline{critical numbers} of the original function $f(x)$, and then use the first or second derivative tests to classify them as local maximums, local minimums, or neither.  Then give the intervals on which $f$ is increasing or decreasing.  

Use the middle of the page to show your work and record your answers at the bottom of the page.
\vfill
Critical numbers:
\vskip 1em
Local maximum(s) at $x = $
\vskip 1em
Local minimum(s) at $x = $
\vskip 1em
$f$ is increasing on the interval(s):
\vskip 1em
$f$ is decreasing on the interval(s):
\end{questions}

\newpage

\def\version{C}
%space for name
\noindent {\large\bf Name:} \underline{\hspace{2.5 in}}
\vskip 1em

\begin{questions}
\question The function $f(x)$ (which you don't know) has \emph{first and second derivatives}
\[
f'(x) = \frac{(x+4)(x-2)}{x-1}
\]
\[
f''(x) = \frac{x^2 - 2x + 6}{(x-1)^2}
\]
Using these provided derivatives, find all \underline{critical numbers} of the original function $f(x)$, and then use the first or second derivative tests to classify them as local maximums, local minimums, or neither.  Then give the intervals on which $f$ is increasing or decreasing.  

Use the middle of the page to show your work and record your answers at the bottom of the page.
\vfill
Critical numbers:
\vskip 1em
Local maximum(s) at $x = $
\vskip 1em
Local minimum(s) at $x = $
\vskip 1em
$f$ is increasing on the interval(s):
\vskip 1em
$f$ is decreasing on the interval(s):
\end{questions}


\newpage

\def\version{D}
%space for name
\noindent {\large\bf Name:} \underline{\hspace{2.5 in}}
\vskip 1em

\begin{questions}
\question The function $f(x)$ (which you don't know) has \emph{first and second derivatives}
\[
f'(x) = (x+2)(x-1)^2 
\]
\[
f''(x) = 3(x-1)(x+1)
\]
Using these provided derivatives, find all \underline{critical numbers} of the original function $f(x)$, and then use the first or second derivative tests to classify them as local maximums, local minimums, or neither.  Then give the intervals on which $f$ is increasing or decreasing.  

Use the middle of the page to show your work and record your answers at the bottom of the page.
\vfill
Critical numbers:
\vskip 1em
Local maximum(s) at $x = $
\vskip 1em
Local minimum(s) at $x = $
\vskip 1em
$f$ is increasing on the interval(s):
\vskip 1em
$f$ is decreasing on the interval(s):
\end{questions}

\end{document}
