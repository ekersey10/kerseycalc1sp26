\documentclass[11pt]{exam}

\usepackage{amssymb, amsmath, amsthm, mathrsfs, multicol, graphicx}
\usepackage{tikz, pgfplots}


\def\d{\displaystyle}
\def\?{\reflectbox{?}}
\def\b#1{\mathbf{#1}}
\def\f#1{\mathfrak #1}
\def\c#1{\mathcal #1}
\def\s#1{\mathscr #1}
\def\r#1{\mathrm{#1}}
\def\N{\mathbb N}
\def\Z{\mathbb Z}
\def\Q{\mathbb Q}
\def\R{\mathbb R}
\def\C{\mathbb C}
\def\F{\mathbb F}
\def\A{\mathbb A}
\def\X{\mathbb X}
\def\E{\mathbb E}
\def\O{\mathbb O}
\def\pow{\mathscr P}
\def\inv{^{-1}}
\def\nrml{\triangleleft}
\def\st{:}
\def\~{\widetilde}
\def\rem{\mathcal R}
\def\iff{\leftrightarrow}
\def\Iff{\Leftrightarrow}
\def\and{\wedge}
\def\And{\bigwedge}
\def\AAnd{\d\bigwedge\mkern-18 mu\bigwedge}
\def\Vee{\bigvee}
\def\VVee{\d\Vee\mkern-18 mu\Vee}
\def\imp{\rightarrow}
\def\Imp{\Rightarrow}
\def\Fi{\Leftarrow}


\def\bar{\overline}

%\pointname{pts}
\pointsinmargin
\marginpointname{pts}
\marginbonuspointname{ bns pts}

\addpoints
\pagestyle{headandfoot}
%\printanswers


\header{MATH 131}{\bf\large Learning Target 7 Quiz}{Fall 2025}
\runningfooter{}{}{Version \version}
\extrafootheight{-.45 in}



\begin{document}
\def\version{A}
%space for name
\noindent {\large\bf Name:} \underline{\hspace{2.5 in}}
\vskip 1em


\begin{questions}
\question Suppose you know that the function $f$ passes through the point $(4,3)$ and has first derivative
\[
f'(x) = \sqrt{x} + 5.
\]
\begin{parts}
\part Find the equation of the tangent line to the the function $f(x)$ at the point $(4,3)$.
\vfill
\vfill
\part Use the tangent line (or the equivalent \emph{local linearization}) to approximate $f(4.1)$.  Show your work.
\vfill
\part Suppose you found out that $f''(4) = 0.25$.  What does this tell you about the shape of $f$ near $x = 4$?  Does this mean your approximation for $f(4.1)$ is an over estimate or under estimate?  Briefly explain.
\vfill
\end{parts}
\end{questions}



\newpage

\def\version{B}
%space for name
\noindent {\large\bf Name:} \underline{\hspace{2.5 in}}
\vskip 1em


\begin{questions}
 \question Suppose you know that the function $f$ passes through the point $(3,1)$ and has first derivative
\[
f'(x) = x^2 - x.
\]
\begin{parts}
\part Find the equation of the tangent line to the the function $f(x)$ at the point $(3,1)$.
\vfill
\vfill
\part Use the tangent line (or the equivalent \emph{local linearization}) to approximate $f(2.9)$.  Show your work.
\vfill
\part Suppose you found out that $f''(3) = 2$.  What does this tell you about the shape of $f$ near $x = 3$?  Does this mean your approximation for $f(3.1)$ is an over estimate or under estimate?  Briefly explain.
\vfill
\end{parts}
\end{questions}

\newpage

\def\version{C}
%space for name
\noindent {\large\bf Name:} \underline{\hspace{2.5 in}}
\vskip 1em

\begin{questions}
  \question Suppose you know that the function $f$ passes through the point $(2,6)$ and has first derivative
\[
f'(x) = \frac{1}{x}.
\]
\begin{parts}
\part Find the equation of the tangent line to the the function $f(x)$ at the point $(2,6)$.
\vfill
\vfill
\part Use the tangent line (or the equivalent \emph{local linearization}) to approximate $f(2.4)$.  Show your work.
\vfill
\part Suppose you found out that $f''(2) = -0.25$.  What does this tell you about the shape of $f$ near $x = 2$?  Does this mean your approximation for $f(2.4)$ is an over estimate or under estimate?  Briefly explain.
\vfill
\end{parts}
\end{questions}


\newpage

\def\version{D}
%space for name
\noindent {\large\bf Name:} \underline{\hspace{2.5 in}}
\vskip 1em


\begin{questions}
 \question Suppose you know that the function $f$ passes through the point $(2,8)$ and has first derivative
\[
f'(x) = 2x - 3x^2.
\]
\begin{parts}
\part Find the equation of the tangent line to the the function $f(x)$ at the point $(2,8)$.
\vfill
\vfill
\part Use the tangent line (or the equivalent \emph{local linearization}) to approximate $f(1.9)$.  Show your work.
\vfill
\part Suppose you found out that $f''(2) = -10$.  What does this tell you about the shape of $f$ near $x = 2$?  Does this mean your approximation for $f(1.9)$ is an over estimate or under estimate?  Briefly explain.
\vfill
\end{parts}
\end{questions}

\end{document}
