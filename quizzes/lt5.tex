\documentclass[11pt]{exam}

\usepackage{amssymb, amsmath, amsthm, mathrsfs, multicol, graphicx}
\usepackage{tikz, pgfplots}


\def\d{\displaystyle}
\def\?{\reflectbox{?}}
\def\b#1{\mathbf{#1}}
\def\f#1{\mathfrak #1}
\def\c#1{\mathcal #1}
\def\s#1{\mathscr #1}
\def\r#1{\mathrm{#1}}
\def\N{\mathbb N}
\def\Z{\mathbb Z}
\def\Q{\mathbb Q}
\def\R{\mathbb R}
\def\C{\mathbb C}
\def\F{\mathbb F}
\def\A{\mathbb A}
\def\X{\mathbb X}
\def\E{\mathbb E}
\def\O{\mathbb O}
\def\pow{\mathscr P}
\def\inv{^{-1}}
\def\nrml{\triangleleft}
\def\st{:}
\def\~{\widetilde}
\def\rem{\mathcal R}
\def\iff{\leftrightarrow}
\def\Iff{\Leftrightarrow}
\def\and{\wedge}
\def\And{\bigwedge}
\def\AAnd{\d\bigwedge\mkern-18 mu\bigwedge}
\def\Vee{\bigvee}
\def\VVee{\d\Vee\mkern-18 mu\Vee}
\def\imp{\rightarrow}
\def\Imp{\Rightarrow}
\def\Fi{\Leftarrow}

\def\bar{\overline}

%\pointname{pts}
\pointsinmargin
\marginpointname{pts}
\marginbonuspointname{ bns pts}

\addpoints
\pagestyle{headandfoot}
%\printanswers


\header{MATH 131}{\bf\large Learning Target 5 Quiz}{Fall 2025}
\runningfooter{}{}{Version \version}
\extrafootheight{-.45 in}



\begin{document}
\def\version{A}
%space for name
\noindent {\large\bf Name:} \underline{\hspace{2.5 in}}
\vskip 1em

\begin{questions}
\question Your baby pygmy hippo's weight, measured in pounds (lbs), is a function of its age, measured in days.  That is, $w(t)$ gives the weight of the hippo $t$ days after birth.

Write a sentence accurately interpreting the two equations below, \emph{including units}.  Your sentence must use all numbers in the equations.
\[
w(30) = 35 \qquad \qquad w'(30) = 2
\]

\vfill

\question You are buying stickers to give out to celebrate your baby pygmy hippo, and the sticker company has offered you a discount if you buy in bulk.  The total cost in dollars is given by a function $c(x)$ for buying $x$ stickers.  

Write a sentence accurately interpreting the two equations below, \emph{including units}.  Your sentence must use all numbers in the equation.
\[
c(100) = 12 \qquad\qquad c'(100) = 0.05
\]

\vfill
\end{questions}



\newpage

\def\version{B}
%space for name
\noindent {\large\bf Name:} \underline{\hspace{2.5 in}}
\vskip 1em

\begin{questions}
 \question The number of minutes of daylight is a function of day of the year.  That is, $d(t)$ gives the how many minutes of daylight there are on day $t$ of the year (since January 1st).

Write a sentence accurately interpreting the two equations below, \emph{including units}.  Your sentence must use all numbers in the equations.
\[
d(260) = 744 \qquad \qquad d'(260) = -2.6
\]

\vfill

\question The  daily energy consumption of an average household in Greeley during the summer is a function of the high temperature that day.  That is, $f(t)$ is the number of kilowatt hours (kWh) used on a day when the hight temperature is $t$ degrees Fahrenheit. 

Write a sentence accurately interpreting the two equations below, \emph{including units}.  Your sentence must use all numbers in the equation.
\[
f(80) = 30 \qquad\qquad f'(80) = 0.25
\]

\vfill
\end{questions}

\newpage

\def\version{C}
%space for name
\noindent {\large\bf Name:} \underline{\hspace{2.5 in}}
\vskip 1em

\begin{questions}
  \question The height of a giraffe is a function of its age.  That is, $h(t)$ gives the height (in inches) of a giraffe $t$ days after it is born.

Write a sentence accurately interpreting the two equations below, \emph{including units}.  Your sentence must use all numbers in the equations.
\[
h(1) = 72 \qquad \qquad h'(1) = 0.2
\]

\vfill

\question The weight of a giraffe is a function of its height: $w(h)$ gives the weight in pounds (lbs) of a giraffe when it is $h$ inches tall.

Write a sentence accurately interpreting the two equations below, \emph{including units}.  Your sentence must use all numbers in the equation.
\[
w(72) = 150 \qquad\qquad w'(72) = 2
\]

\vfill
\end{questions}


\newpage

\def\version{D}
%space for name
\noindent {\large\bf Name:} \underline{\hspace{2.5 in}}
\vskip 1em

\begin{questions}
 \question The population of Greeley is a function of time, measured in years since 2000.  That is, $p(t)$ gives the population $t$ years after 2000.

Write a sentence accurately interpreting the two equations below, \emph{including units}.  Your sentence must use all numbers in the equations.
\[
p(24) = 114,363 \qquad \qquad p'(24) = 2,000
\]

\vfill

\question To weekly water consumption in Greeley is a function of its population: $w(p)$ gives the number of gallons used in a week when the population is $p$ people.

Write a sentence accurately interpreting the two equations below, \emph{including units}.  Your sentence must use all numbers in the equation.
\[
w(100,000) = 70,000,000 \qquad\qquad w'(100,000) = 600
\]

\vfill
\end{questions}

\end{document}
